% Created 2023-04-27 Thu 13:16
% Intended LaTeX compiler: pdflatex
\documentclass[11pt]{article}
\usepackage[utf8]{inputenc}
\usepackage[T1]{fontenc}
\usepackage{graphicx}
\usepackage{longtable}
\usepackage{wrapfig}
\usepackage{rotating}
\usepackage[normalem]{ulem}
\usepackage{amsmath}
\usepackage{amssymb}
\usepackage{capt-of}
\usepackage{hyperref}
\author{Yang Huang}
\date{\textit{<2023-01-18 Wed>}}
\title{Vibrational Entropy of Crystalline Solids from Covariance of Atomic Displacements}
\hypersetup{
 pdfauthor={Yang Huang},
 pdftitle={Vibrational Entropy of Crystalline Solids from Covariance of Atomic Displacements},
 pdfkeywords={},
 pdfsubject={},
 pdfcreator={Emacs 25.2.2 (Org mode 9.2.3)}, 
 pdflang={English}}
\begin{document}

\maketitle
\tableofcontents


\section{Purpose}
\label{sec:orgdec938a}
This archive contains tools to calculate entropy from covariance matrix based
on method describe in Huang, Y.; Widom, M. Vibrational Entropy of Crystalline
Solids from Covariance of Atomic Displacements. Entropy 2022, 24, 618.
\url{https://doi.org/10.3390/e24050618}.
These tools are designed especially for output from MD simulation implemented
in VASP. You are free to modify and redistribute the code for general purpose.

\section{Contents}
\label{sec:org6d808f0}
Following list of files should be contained in the directory:
\begin{itemize}
\item README

\item install.sh

\item src

Contain source files.

\item example

Contain examples

\item bin

Contain compiled binaries.
\end{itemize}


\section{Required packages}
\label{sec:org87dc4ed}
\begin{enumerate}
\item GSL - GNU Scientific Library

The GNU Scientific Library (GSL) is a numerical library for C and C++
programmers. It is free software under the GNU General Public License. 
It is installed on most linux systems. To install it, check \url{https://www.gnu.org/software/gsl/}.

\item Aflow [optional]

The aflow-sym.sh requires AFLOW codes in order to run properly. 
To install it, check \url{http://aflow.org/install-aflow/}.

\item pos2xyz and xyz2pos from euler

The xdat-ent program is originally written on euler so that it utilizes certain
mgtools on euler. Unfortunately the author is unwilling to modify
these codes to make it more convenient for general users and you will have to
ask for permissions to access these tools.

The "pos2xyz" converts a POSCAR file to a XYZ file.
The "xyz2pos" converts a XYZ file to a POSCAR file.
\end{enumerate}

\section{How to install}
\label{sec:org8ee82fe}
Go to the main directory and run "./install.sh". It will compile the source
files and put binaries into the bin directory.

After you run the command, you run the test script "./test.sh" in the examples directory.

\section{List of tools}
\label{sec:org1e91d21}
[BEFORE YOU RUN]
\begin{itemize}
\item xdat-ent-init.sh

\begin{itemize}
\item Description

Preparing "Trun" and "Mass" input files for "xdat-ent-rhoALL\_quantum\_mod"
commands.

\item Input:
\begin{itemize}
\item INCAR
\item POTCAR
\item /home/Tools/src/mgtools/INFO/CHTAB.
\end{itemize}

\item Usage:

xdat-ent-init.sh
*Only work on euler.

\item Output:

Generate "Trun" and "Mass".
\end{itemize}

\item aflow-sym.sh
\begin{itemize}
\item Description

This script utilizes "aflow" tools to help find symmetry operations. It is recommended to
run on a small cell for efficiency.

\item Input:

POSCAR or XYZ file.

\item Usage:

aflow-sym.sh [ -x <xyz file> ] [ -p <POSCAR file> ]

\item Output:

A "symmetry\_operation.dat " which contains number of operations N and N
3x3 operation matrix in direction space.
\end{itemize}
\end{itemize}

[WHEN YOU RUN]
\begin{itemize}
\item xdat-ent-rhoALL\_quantum\_mod tools

\begin{itemize}
\item Description

\begin{itemize}
\item xdat-ent-rhoALL\_quantum\_mod

Pure element crystalline from NVT simulation

\item xdat-ent-rhoALL\_quantum\_npt

Pure element crystalline from NPT simulation

\item xdat-ent-rhoALL\_quantum\_alloy

Alloys from NVT simulation.
\end{itemize}

xdat-ent-rhoALL\_quantum\_mod has been tested to be robust while only a few
examples have been performed with xdat-ent-rhoALL\_quantum\_npt and
xdat-ent-rhoALL\_quantum\_alloy command. So be careful with these two
commands!

\item Input:

"example" directory provides examples of input files.

\begin{itemize}
\item Trun

This file contains in value which is the temperature of MD simulation.

\item Mass

This file contains three rows:
\begin{enumerate}
\item type of species.
\item atomic numbers.
\item atomic masses [a.u.] in atomic units.
\end{enumerate}

\item xyzfile

This file contains the ideal structure of a crystal.
The first three rows are lattice constants.
The 4th row contains number of atoms N.
The following N rows should obeys:
dx,dy,dz,na \ldots{}
where dx,dy,dz is directional vector and na is atomic number.

\item symmetry-file

This file contains symmetry operation matrices U.
The first row is number of operations N and then it is followed by N 3x3
matrices U each of which is a operation matrix in directional space (b'=Ub).

\item XDATCAR

Output from VASP MD simulations.
\end{itemize}
\end{itemize}
\end{itemize}


\begin{itemize}
\item Usage:

xdat-ent-rhoALL\_quantum\_mod -f xyzfile -s symmetry-file XDATCAR* > out

\item Output:

\begin{itemize}
\item covariance\_pair.dat

3x3 covariance matrices in real space of all symmetry distinguishing
pairs.

\item density\_matrix.dat

See reference for more mathematical description.
A 3Nx3N matrix.

\item force\_matrix.dat

See reference for more mathematical description.
A 3Nx3N force matrix.

\item Eigenvalues.dat

A 3Nx2 matrix.
The first row are eigenvalues of density matrix.
The second row are frequencies (hbar*omega [eV]) of eigen modes
See reference for mathematical relation.

\item "out":

The last line of "out" reports calculated entropy with S\_classic
(classical entropy) and S\_quantum (quantum entropy)
\end{itemize}
\end{itemize}

[AFTER YOU RUN] 
\begin{itemize}
\item freq2quan.py

\begin{itemize}
\item Description

Calculate free energy F[eV/at], internal energy U [eV/at], entropy S [kB/at] 
and heat capacity Cv [kB/at] from Eigenvalues.dat

\item Input: "Trun", "Eigenvalues.dat"

\item Usage: freq2quan.py

\item Output: calculated quantities.
\end{itemize}

\item force2dos

\begin{itemize}
\item Description

Calculate free energy F[eV/at], internal energy U [eV/at], entropy S [kB/at] 
and heat capacity Cv [kB/at] from Eigenvalues.dat

\item Input: 

\begin{itemize}
\item "Trun", "force\_matrix.dat", xyzfile

\item KDim (an integer indicate kmesh KDimxKDimxKdim).

\item recip\_vectors of a primitive cell

3x3 matrix
\end{itemize}

\item Usage: 

force2dos xyzfile KDim > out;

\item Output: 

In "out" file, there are two columns of datas.
The first column is hbar*omega [meV].
The second one is vibrational DOS [states/atom/meV].
\end{itemize}

\item force2band

\begin{itemize}
\item Description

Calculate free energy F[eV/at], internal energy U [eV/at], entropy S [kB/at] 
and heat capacity Cv [kB/at] from Eigenvalues.dat

\item Input: 

\begin{itemize}
\item "Trun", "force\_matrix.dat", xyzfile

\item KDim (an integer indicate kmesh KDimxKDimxKdim).

\item recip\_vectors of a primitive cell

3x3 matrix

\item kpath
\end{itemize}

\item Usage: 

force2band xyzfile > out;

\item Output: 

Eigenvalues along kpath.
\end{itemize}
\end{itemize}

\section{Pay attention to}
\label{sec:org8888f66}

\begin{itemize}
\item symmetry operations:

Symmetry operations will greatly improve precision of calculated entropy
for a short run at a cost of calculation time. With symmetry operation
turned on, each configuration will be process N times where N is the number
of operations.

This program is known to give incorrect results when applied to low-symmetry super
cell from a highly symmetrical crystalline structure, e.g. a 5x5x5 primitive
cell of BCC Li. Hence, it is recommended to run simulation at its
high-symmetry unit cell. e.g. a super cell of a 2-atom BCC unit cell or a 4-atom FCC
unit cell.

\item Eigenvalues.dat:

A reasonable calculation should give 3 zero eigenvalues in the first column
in Eigenvalues.dat and correspondingly 3 nan or large number in the second
column. So be careful with zero or nan values in Eigenvalues.dat.

Imaginary modes may show up which are denoted as negative eigenvalues. As
long as there are 3 zero eigenvalues, we think the program is doing its job.
\end{itemize}
\end{document}